
\documentclass[tikz,border=3.14mm]{standalone}
\usepackage{pgfplots}
\pgfplotsset{compat=1.18}
\usepackage{mathtools}
\usetikzlibrary{angles,quotes,arrows.meta}

% Line from origin passing through a point 
% #1=x:
% #2=y:
\pgfmathdeclarefunction{oline}{2}{%
  \pgfmathparse{(#2/#1)*x}%
}

\begin{document}

\begin{tikzpicture}
\begin{axis}[
  no markers,
  domain=0:10,
  samples=1000,
  axis lines*=left,
  y axis line style={draw=none},
  xlabel=$$,
  ylabel=$$,
  every axis y label/.style={
    at=(current axis.above origin),
    anchor=south
  },
  every axis x label/.style={
    at=(current axis.right of origin),
    anchor=west
  },
  height=3cm,
  width=10cm,
  xtick={0, 5},
  ytick=\empty,
  xticklabels={\strut 0, \strut t},
  enlargelimits=false,
  clip=false,
  axis on top,
  grid = major,
]
  \addplot[thick, blue]
    {oline(5,1)};

  % % Shading part of the first plot
  % \addplot[thick, blue]
  %   {oline(5,1)};

  % Shading part of the first plot
  \addplot[fill=gray!20,
           draw=none,
          ]
    {oline(5,1)}
    \closedcycle;

  \addplot[thick, red!50!black] {x>=5 ? 1 : 0};
\end{axis}
\end{tikzpicture}
\end{document}
