\documentclass[12pt]{article}

\usepackage{amsmath}
\usepackage{amssymb}
\usepackage{stmaryrd}
\usepackage{booktabs}
\usepackage{mathpartir}
\usepackage[backend=bibtex, style=numeric]{biblatex}
\usepackage{hyperref}
\usepackage{soul}

% https://tex.stackexchange.com/questions/611251/in-circuitikz-multiwire-how-to-so-specify-label-placement
\usepackage[siunitx, RPvoltages, fulldiode, american voltages]{circuitikz}
\newcommand{\mwire}[3]{% args: position (0.1-0.9), label, final coordinate
    coordinate(tmp1) #3 coordinate(tmp2)
    (tmp1) -- ($(tmp1)!{#1-0.1}!(tmp2)$) to[multiwire=#2] ($(tmp1)!{#1+0.1}!(tmp2)$) -- (tmp2)
}

\usepackage{tikz}
\usetikzlibrary{positioning}
\usetikzlibrary{calc,decorations.markings}
\usetikzlibrary{quantikz2}
\usetikzlibrary{external}
\tikzexternalize

\usetikzlibrary{positioning}
\ctikzset{logic ports=ieee}

% \newcommand\ket[1]{
%   | #1 \rangle
% }

%   % \left\langle #1 \middle| #2 \right\rangle
% \newcommand\bra[1]{
%   \langle #1 |
% }

\newcommand\kket[1]{
  | #1 \rangle \rangle
}
% | #1 \rangle \kern-1.35pt \rangle


\begin{document}

\title{Lecture 3: The Walsh-Hadamard transform, Deutsch algorithm and the Deutsch-Josza problem}
\author{Lecturer: Dinesh Krishnamoorthy\\ Scribe: Julin Shaji}
\date{June 11, 2025}
\maketitle

\section{Quiz 5 discussion}
What's the output state of this gate?

\begin{center}
\begin{circuitikz} 
  \tikzset{%
    mux33/.style={
      muxdemux,
      muxdemux def={Lh=5, Rh=5, NL=3, NR=3, NB=0},%
      % muxdemux label={%
      %     L1=$\ket{x}$, L2=$\ket{y}$, L3=$\ket{0}$,
      %     R1=$\ket{x}$, R2=$\ket{y}$, R3=},
    }%
  }
  \node[mux33] (G) {T};
  \draw (G.lpin 1) ++(-0.2, 0) node[xshift=-2mm] {$\ket{x}$};
  \draw (G.lpin 2) ++(-0.2, 0) node[xshift=-2mm] {$\ket{y}$};
  \draw (G.lpin 3) ++(-0.2, 0) node[xshift=-2mm] {$\ket{-}$};

  \draw (G.rpin 1) ++(0.2, 0) node[xshift=2mm] {?};
  \draw (G.rpin 2) ++(0.2, 0) node[xshift=2mm] {?};
  \draw (G.rpin 3) ++(0.2, 0) node[xshift=2mm] {?};
\end{circuitikz} 
\end{center}

\begin{mathpar}
  \ket{-} = H\ket{1} = \frac{1}{\sqrt{2}}\ket{0} - \frac{1}{\sqrt{2}}\ket{1}
\end{mathpar}

Toffoli gate is like:

\begin{center}
\begin{circuitikz} 
  \tikzset{%
    mux33/.style={
      muxdemux,
      muxdemux def={Lh=5, Rh=5, NL=3, NR=3, NB=0},%
    }
  }
  \node[mux33] (G) {T};
  \draw (G.lpin 1) ++(-0.2, 0) node[xshift=-2mm] {x};
  \draw (G.lpin 2) ++(-0.2, 0) node[xshift=-2mm] {y};
  \draw (G.lpin 3) ++(-0.2, 0) node[xshift=-2mm] {z};

  \draw (G.rpin 1) ++(0.2, 0) node[xshift=2mm] {x};
  \draw (G.rpin 2) ++(0.2, 0) node[xshift=2mm] {y};
  \draw (G.rpin 3) ++(0.2, 0) node[xshift=6.5mm] {z $\oplus$ xy};
\end{circuitikz} 
\end{center}

So,

\begin{mathpar}
  \begin{array}{cl}
  & T(\ket{xy}\ket{-}) \\
= & \ket{xy} T(\ket{-}) \\
= & \ket{xy}\left(\ket{-} \oplus xy \right)  \\
= & \ket{xy}\left( \frac{1}{\sqrt{2}}\ket{0 \oplus xy} - \frac{1}{\sqrt{2}}\ket{1 \oplus xy}\right) \\
  \end{array}
  
  \begin{array}{cl}
  & T(\ket{00}\ket{-}) \\
= & \ket{00}\left( \frac{1}{\sqrt{2}}\ket{0 \oplus 0.0} - \frac{1}{\sqrt{2}}\ket{1 \oplus 0.0}\right) \\
= & \ket{00}\left( \frac{1}{\sqrt{2}}\ket{0} - \frac{1}{\sqrt{2}}\ket{1}\right) \\
= & \ket{00}\ket{-} \\
  \end{array}
  
  \begin{array}{cl}
  & T(\ket{01}\ket{-}) \\
= & \ket{01}\left( \frac{1}{\sqrt{2}}\ket{0 \oplus 0.1} - \frac{1}{\sqrt{2}}\ket{1 \oplus 0.1}\right) \\
= & \ket{01}\left( \frac{1}{\sqrt{2}}\ket{0} - \frac{1}{\sqrt{2}}\ket{1}\right) \\
= & \ket{00}\ket{-} \\
  \end{array}
  
  \begin{array}{cl}
  & T(\ket{10}\ket{-}) \\
= & \ket{10}\left( \frac{1}{\sqrt{2}}\ket{0 \oplus 1.0} - \frac{1}{\sqrt{2}}\ket{1 \oplus 1.0}\right) \\
= & \ket{10}\left( \frac{1}{\sqrt{2}}\ket{0} - \frac{1}{\sqrt{2}}\ket{1}\right) \\
= & \ket{00}\ket{-} \\
  \end{array}
  
  \begin{array}{cl}
  & T(\ket{11}\ket{-}) \\
= & \ket{11}\left( \frac{1}{\sqrt{2}}\ket{0 \oplus 1.1} - \frac{1}{\sqrt{2}}\ket{1 \oplus 1.1}\right) \\
= & \ket{11}\left( \frac{1}{\sqrt{2}}\ket{1} - \frac{1}{\sqrt{2}}\ket{0}\right) \\
= & -\ket{00}\ket{-} \\
  \end{array}
\end{mathpar}

Could think of $f(x, y) =  xy$ and output state being

\begin{mathpar}
  T(\ket{xy}\ket{-}) = (-1)^{xy}\ket{xy}\ket{-}
\end{mathpar}

Representation where function value is coming together with amplitude:
\emph{Phased representation}


\begin{itemize}
\item Right now, we got the second input and second ouput lying around
\item Can we make the gate interface simpler by hiding them?
\item Like:

  \begin{circuitikz} 
    \tikzset{%
      mux11/.style={
        muxdemux,
        muxdemux def={Lh=3, Rh=3, NL=1, NR=1, NB=0},%
      },%
      buswidth/.style={decoration={
        markings,
        mark= at position 0.5 with {\node[font=\footnotesize] {/};\node[below=1pt] {\tiny #1};}
        }, postaction={decorate}}
    }
    \node[mux11] (G) {$U_f^{\pm}$};
    \draw (G.lpin 1) \mwire{0.2}{n}{++(-0.2, 0)} node[xshift=-2mm] {$\ket{x}$};
    \draw (G.rpin 1) \mwire{0.2}{n}{++(0.2, 0)} node[xshift=9mm] {$(-1)^{f(x)}\ket{x}$};

    % \draw[buswidth={n}] (G.lpin 1) ++(-0.2, 0) node[xshift=-2mm] {$\ket{x}$};
    % \draw[buswidth={n}] (G.rpin 1) ++(0.2, 0) node[xshift=9mm] {$(-1)^{f(x)}\ket{x}$};
  \end{circuitikz} 

\end{itemize}

\begin{mathpar}
  
  \begin{array}{cl}
  & U_f(\ket{x}\ket{-}) \\
= & U_f(\ket{x}\left(\frac{1}{\sqrt{2}}\ket{0} - \frac{1}{\sqrt{2}}\ket{1} \right) \\
= & 

  \end{array}
\end{mathpar}


$s$ is the secret.

\begin{quantikz}
 \lstick{$\kket{s}$} &           &          & \ctrl{1}  \\
 \lstick{$\kket{0}$} & \gate{\$} & \ctrl{2} & \targ{}   \\
 \lstick{$\kket{0}$} &           & \targ{}  &
\end{quantikz}

\section{Deutsch's problem}
\begin{itemize}
\item Given a Boolean function $f: \{01\} \mapsto \{01\}$
\item Four possibilities ($2*2 = 4$)
\item $f$ is a \emph{constant function} if $\forall x, f(x)=0$ or $\forall x, f(x)=1$
\item $f$ is a \emph{balanced function} if number of f(x)=0 and f(x)=1 are
  the same
\item 
\textbf{Problem}:
Assuming that $f$ is either balanced or constant, find which among
them two $f$ is with the least number of queries to the truth table of
$f$.
\item ie, in the case of this $f$, compute $f(0) \oplus f(1)$
  \begin{itemize}
  \item $f(0) \oplus f(1) = 0 \Rightarrow$ constant function
  \item $f(0) \oplus f(1) = 1 \Rightarrow$ balanced function
  \end{itemize}
\end{itemize}


\section{Deutsch-Josza problem}
\begin{itemize}
\item A generalization of Deutsch's problem
\item Given a Boolean function $f: \{01\}^n \mapsto \{01\}$
\item Find if $f$ is constant or balanced when it is known that $f$ is
  one of the two.
\item With least queries to $f$
\item This was the first algorithm for which a quantum algorithm was
  found to be better than the best classical algorithm
\item Classical algorithm $\Rightarrow$ has to query at least
  $\frac{n}{2}+1$ times before a decision can be made
  \begin{itemize}
  \item Look at $\frac{n}{2}+1$ random unique entries of $f$
  \item If it's either all $0$ or all $1$, then it's constant
  \item Otherwise it's balanced
  \item Because it's already known that $f$ is either constant or balanced
  \end{itemize}
\item Quantum algorithm $\Rightarrow$ only one query needed (because
  of superpositioned states)
\item Balanced $\Rightarrow$ $2^{n-1}$ are $0$, $2^{n-1}$ are $1$
\end{itemize}

The quantum algorithm $\Rightarrow$ input states are in superposition.

\end{document}
