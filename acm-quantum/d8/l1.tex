\documentclass[12pt]{article}

\usepackage{amsmath}
\usepackage{amssymb}
\usepackage{stmaryrd}
\usepackage{booktabs}
\usepackage{mathpartir}
\usepackage[backend=bibtex, style=numeric]{biblatex}
\usepackage{hyperref}
\usepackage{soul}

% https://tex.stackexchange.com/questions/611251/in-circuitikz-multiwire-how-to-so-specify-label-placement
\usepackage[siunitx, RPvoltages, fulldiode, american voltages]{circuitikz}
\newcommand{\mwire}[3]{% args: position (0.1-0.9), label, final coordinate
    coordinate(tmp1) #3 coordinate(tmp2)
    (tmp1) -- ($(tmp1)!{#1-0.1}!(tmp2)$) to[multiwire=#2] ($(tmp1)!{#1+0.1}!(tmp2)$) -- (tmp2)
}

\usepackage{tikz}
\usetikzlibrary{positioning}
\usetikzlibrary{calc,decorations.markings}
\usetikzlibrary{quantikz2}
\usetikzlibrary{external}
\tikzexternalize

\usetikzlibrary{positioning}
\ctikzset{logic ports=ieee}

% \newcommand\ket[1]{
%   | #1 \rangle
% }

%   % \left\langle #1 \middle| #2 \right\rangle
% \newcommand\bra[1]{
%   \langle #1 |
% }

\newcommand\kket[1]{
  | #1 \rangle \rangle
}
% | #1 \rangle \kern-1.35pt \rangle


\begin{document}

\title{Lecture 1: Quantum Fourier transform}
\author{Lecturer: Rajendra Kumar\\ Scribe: Julin Shaji}
\date{June 17, 2025}
\maketitle

\section{Intro}
\begin{itemize}
\item Fourier transform $\Rightarrow F_N$
\item $F_N$ is unitary $\Rightarrow$ can use quantum computing
\item $FT(v) = \hat{v} = F_N \cdot v$
\item Classical FT $\Rightarrow$ we can get $\hat{v}$ completely
\item
  But quantum $\Rightarrow$ output is a superposition state over
  amplitudes of all states
\end{itemize}


\section{Rotation gates ($R_k$)}

\begin{mathpar}
R_k = 
  \begin{bmatrix}
    1 & 0 \\
    0 & e^{2 \pi i / 2^s} \\
  \end{bmatrix}
\end{mathpar}

which means

\begin{mathpar}
R_0 = % 
  \begin{bmatrix}
    1 & 0 \\
    0 & e^{2 \pi i / 2^0}
  \end{bmatrix} = %
  \begin{bmatrix}
    1 & 0 \\
    0 & 1
  \end{bmatrix} = I
\\
R_1 = % 
  \begin{bmatrix}
    1 & 0 \\
    0 & e^{2 \pi i / 2^1}
  \end{bmatrix} = %
  \begin{bmatrix}
    1 & 0 \\
    0 & -1
  \end{bmatrix} = Z
\\
R_2 = % 
  \begin{bmatrix}
    1 & 0 \\
    0 & e^{2 \pi i / 2^2}
  \end{bmatrix} = %
  \begin{bmatrix}
    1 & 0 \\
    0 & i
  \end{bmatrix} = Y ??
\end{mathpar}

Because:
\begin{itemize}
\item $e^{2\pi i t} = 1, t \in \mathbb{Z}$
\item $e^{\pi i} = -1$
\item $e^{\pi i/2} = i$
\end{itemize}

Note that $s \to \infty$ implies $R_k \to I$ (identity matrix)

\section{QFT}
Let:
\begin{itemize}
\item $v$: an $N$-qubit state
\item $N = 2^n$ which means $2^n$ states
\item $\ket{k} = \ket{k_1 k_2 \cdots k_n}$
\item $j$ is a number whose bit-representation is $j_1 j_2 \cdots j_n$
\end{itemize}

\textbf{Fun fact}: QFT on $\ket{0^n}$ is same as taking $H^{\otimes n}$ on $\ket{0^n}$.

\subsection{Inverse QFT}
\begin{itemize}
\item Inverse of an unitary operation = its complex conjugate
\item Also, circuit for IQFT is the circuit for QFT applied in reverse order
\end{itemize}

\begin{align*}
  R_k * R_k &=
  \begin{bmatrix}
    1 & 0 \\
    0 & e^{2\pi i / 2^k} \\
  \end{bmatrix} *
  \begin{bmatrix}
    1 & 0 \\
    0 & e^{2\pi i / 2^k} \\
  \end{bmatrix} \\
  &=
  \begin{bmatrix}
    1 & 0 \\
    0 & 1 \\
  \end{bmatrix} \\
\end{align*}

\section{Applying QFT: Period finding problem}
\begin{itemize}
\item 
  Given:
  \begin{itemize}
  \item A function $f: 2^l \to 2^n$
  \item $\exists r>0, f(x) = f(x+r)$
  \item ie, $r$ is the period of $f$
  \end{itemize}
\item Aim: Find $r$
\end{itemize}


\section{Reserve}

\begin{center}
  \begin{quantikz}
    \lstick{\ket{k_1}}
    & \gate{H}
    & \gate{R_2}
    & \gate{R_3}
    &
    &
    &
    & \swap{2}
    & \rstick{} \\
    \lstick{\ket{k_2}}
    &
    & \ctrl{-1}
    & 
    & \gate{H}
    & \gate{R_2}
    & 
    & 
    &  \\
    \lstick{\ket{k_3}}
    & \gate{R_n}
    & 
    & \ctrl{-2}
    & 
    & \ctrl{-1}
    & \gate{H}
    & \targX{} 
    &  \\
  \end{quantikz}
\end{center}

\begin{center}
  \begin{quantikz}
    \lstick[wires=4]{\ket{0^l}}
    & \gate[4]{H^{\otimes l}}
      \slice[label style={pos=1, anchor=north}]{1}
    & \gate[8]{O_f}
      \slice[label style={pos=1, anchor=north}]{2}
    & \gate[4]{F_{2^l}}
      \slice[label style={pos=1, anchor=north}]{3}
    & \meter[4]{}
      \slice[label style={pos=1, anchor=north}]{4}
    & \\
    % \lstick{$\vdots$}
    &
    &
    &
    &
    &  \\
    % \lstick{$\vdots$}
    &
    &
    &
    &
    &  \\
    % \lstick{\ket{0}}
    &
    &
    &
    &
    &  \\
    %
    \lstick[wires=4]{\ket{0^n}}
    % \lstick{\ket{0}}
    & 
    &
    & \meter[4]{}
    &
    &  \\
    % \lstick{$\vdots$}
    &
    &
    &
    &
    &  \\
    % \lstick{$\vdots$}
    &
    &
    &
    &
    &  \\
    % \lstick{\ket{0}}
    &
    &
    &
    &
    &
  \end{quantikz}
\end{center}

\end{document}
