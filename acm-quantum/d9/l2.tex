\documentclass[12pt]{article}

\usepackage{amsmath}
\usepackage{amssymb}
\usepackage{stmaryrd}
\usepackage{booktabs}
\usepackage{mathpartir}
\usepackage[backend=bibtex, style=numeric]{biblatex}
\usepackage{hyperref}
\usepackage{soul}

% https://tex.stackexchange.com/questions/611251/in-circuitikz-multiwire-how-to-so-specify-label-placement
\usepackage[siunitx, RPvoltages, fulldiode, american voltages]{circuitikz}
\newcommand{\mwire}[3]{% args: position (0.1-0.9), label, final coordinate
    coordinate(tmp1) #3 coordinate(tmp2)
    (tmp1) -- ($(tmp1)!{#1-0.1}!(tmp2)$) to[multiwire=#2] ($(tmp1)!{#1+0.1}!(tmp2)$) -- (tmp2)
}

\usepackage{tikz}
\usetikzlibrary{positioning}
\usetikzlibrary{calc,decorations.markings}
\usetikzlibrary{quantikz2}
\usetikzlibrary{external}
\tikzexternalize

\usetikzlibrary{positioning}
\ctikzset{logic ports=ieee}

% \newcommand\ket[1]{
%   | #1 \rangle
% }

%   % \left\langle #1 \middle| #2 \right\rangle
% \newcommand\bra[1]{
%   \langle #1 |
% }

\newcommand\kket[1]{
  | #1 \rangle \rangle
}
% | #1 \rangle \kern-1.35pt \rangle

% \usepackage{amssymb,amsmath,amsthm}
% \usepackage{tikz-cd}
% \usepackage{mathpartir}
% % geometry (sets margin) and other useful packages
% \usepackage[margin=1.25in]{geometry}
% \usepackage{graphicx,ctable,booktabs}
%
% \usepackage[sort&compress,square,comma,authoryear]{natbib}
% \bibliographystyle{plainnat}
%
% \newtheorem{theorem}{Theorem}
% \newtheorem{lemma}{Lemma}

%
%Fancy-header package to modify header/page numbering
%
% \usepackage{fancyhdr}
% \pagestyle{fancy}
% %\addtolength{\headwidth}{\marginparsep} %these change header-rule width
% %\addtolength{\headwidth}{\marginparwidth}
% \lhead{Problem \thesection}
% \chead{}
% \rhead{\thepage}
% \lfoot{\small\scshape EECS 598: Category Theory}
% \cfoot{}
% \rfoot{\footnotesize PS 1}
% \renewcommand{\headrulewidth}{.3pt}
% \renewcommand{\footrulewidth}{.3pt}
% \setlength\voffset{-0.25in}
% \setlength\textheight{648pt}

%%%%%%%%%%%%%%%%%%%%%%%%%%%%%%%%%%%%%%%%%%%%%%%
\begin{document}

\title{Lecture 2: Applications of entanglement}
\author{Lecturer: Debajyoti Bera\\ Scribe: Julin Shaji}
\date{June 18, 2025}
\maketitle

\section{Levels of quantum data}
Apparently, there is no standard name for qudits with $d > 3$.
\footnote{\url{https://quantumcomputing.stackexchange.com/questions/2100/do-any-specific-types-of-qudits-other-than-qubits-and-qutrits-have-a-name}}

\begin{center}
\begin{tabular}{lll}
\toprule
Name   & Num of levels & Possible Values            \\
\midrule
Qubit  & 2             & $\{0, 1\}$                 \\
Qutrit & 3             & $\{0, 1, 2\}$              \\
Qudit  & d             & $\{0, 1, 2, \cdots, d-1\}$ \\
\bottomrule
\end{tabular}
\end{center}

\section{Conjugate coding}
\begin{itemize}
\item Standard basis is $\{\ket{0}, \ket{1}\}$
\item DBT: Z-axis is called so because it corresponds to Pauli gates ??
\end{itemize}

\begin{center}
\begin{tabular}{ll}
  \toprule
  Basis             & Name   \\
  \midrule
$\ket{0}, \ket{1}$  & Z-axis \\
$\ket{+}, \ket{-}$  & X-axis \\
$\ket{i}, \ket{-i}$ & Y-axis \\
  \bottomrule
\end{tabular}
\end{center}

\begin{minipage}{0.5\linewidth}
X axis:
\begin{mathpar}
\ket{+} = \frac{1}{\sqrt{2}}\left( \ket{0} + \ket{1} \right)\\
\ket{-} = \frac{1}{\sqrt{2}}\left( \ket{0} - \ket{1} \right)
\end{mathpar}
\end{minipage}
%
\begin{minipage}{0.5\linewidth}
Y axis:
  \begin{mathpar}
\ket{i} = \frac{1}{\sqrt{2}}\left( \ket{0} + i\ket{1} \right)\\
\ket{-i} = \frac{1}{\sqrt{2}}\left( \ket{0} - i\ket{1} \right)
  \end{mathpar}
\end{minipage}

Inner product of basis from different axes:
DBT: How is this calc done?

\begin{mathpar}
  \begin{array}{rcl}
|\braket{0}{-i}|^2 & = & \frac{1}{\sqrt{2}}^2 \\
                   & = & \frac{1}{2}
  \end{array}
  \\
  \begin{array}{rcl}
|\braket{-}{i}|^2 & = & \frac{1}{\sqrt{2}}\left[ \ket{0} - \ket{1} \right] \frac{1}{\sqrt{2}}\left[ \ket{0} + i\ket{1} \right]\\
                  & = & \frac{1}{2}
  \end{array}
  \\
  \begin{array}{rcl}
|\braket{0}{i}|^2 & = & \frac{1}{2} \\
  \end{array}
\end{mathpar}

\begin{itemize}
\item Bases with this property = mutually unbiased basis
\item aka conjugate bases
\end{itemize}

Two bases:
\begin{itemize}
\item $\{ \ket{a_1}, \ket{a_2} \}$
\item $\{ \ket{b_1}, \ket{b_2} \}$
\end{itemize} are mutually unbiased if
$|\braket{a_1}{b_1}|^2 = \frac{1}{2}$ and 
$|\braket{a_2}{b_2}|^2 = \frac{1}{2}$

ie,

\begin{mathpar}
\forall i j,
|\braket{a_i}{b_j}|^2 = \frac{1}{2} 
\end{mathpar}

\begin{itemize}
\item How large a set of conjugate basis can we come up with?
\item DBT: A hard problem
\item $M(d)$ = maximum number of conjugage axes when $d$ is the number of levels
\item If $d=p^n$ for some prime number $p$ and integer $n$, $M(d)=d+1$
\item d=2 $\Rightarrow$ M(2) = 3 (Z, X and Y axes)
\item M(4) = 5
\item Conjugate bases = $b_1$ and $b_2$
\item DBT: If we have a state in basis $b_1$, the likelihood of getting
  $b_2\ket{+}$ or $b_2\ket{-}$ is $\frac{1}{2}$ each
\item $\Rightarrow$ maximum confusion ??
\end{itemize}

\section{Oblivious transfer}
The problem:
\begin{itemize}
\item Alice has 2 bits: $m_0$, $m_1$
\item Bob wants one of these bits
\item But Bob doesn't want Alice to know which bit is it that he wants
\item And Alice doesn't want Bob to know a bit other than what Bob wants
\item ie, if Bob wants $m_b$ from Alice ($b \in \{0, 1\}$), Bob should
  not be able to figure out $m_{1-b}$
\end{itemize}

Mutually unbiased bases can be used for this.
\begin{itemize}
\item Let bits with Alice be $m_x$ and $m_z$
\item Alice will prepare an initial state = $\ket{\psi}$
\item Alice randomly chooses a basis (or bit) $a$. $a \in_rand \{X, Y\}$
\item Depending on the value of $a$, Alice sends either $m_x$ or $m_z$
  to Bob such that
  \begin{mathpar}
    \ket{\psi} = 
    \begin{cases}
      \ket{+} & \text{if $a=x$, $m_x=0$} \\ 
      \ket{-} & \text{if $a=x$, $m_x=1$} \\ 
      \ket{0} & \text{if $a=z$, $m_x=0$} \\ 
      \ket{1} & \text{if $a=z$, $m_x=1$} \\ 
    \end{cases}
  \end{mathpar}
\end{itemize}

Moving over to Bob's side:
\begin{itemize}
\item If Bob wants $m_x$, Bob will measure in X basis
  \begin{mathpar}
    \begin{cases}
      0 & \text{measure $\ket{+}$} \\
      1 & \text{measure $\ket{-}$} \\
    \end{cases}
  \end{mathpar}
\item If Bob wants $m_z$, Bob will measure in Z basis
  \begin{mathpar}
    \begin{cases}
      0 & \text{measure $\ket{0}$} \\
      1 & \text{measure $\ket{1}$} \\
    \end{cases}
  \end{mathpar}
\item Let $m_{\tilde{b}}$ be the bit that Bob gets
\item Let $m_x=0$.
  \begin{mathpar}
    \begin{array}{rcl}
      P(m_{\tilde{b}}=0) & = & P(\ket{\psi} = \ket{+}) \\
                         & = & \frac{1}{2}*1 + \frac{1}{2}\frac{1}{2} \\
                         & = & \frac{3}{4} \\
                     
    \end{array}
  \end{mathpar}
\item DBT: Sir also said something about the probability being
  $\frac{1}{2}$. What was that?
\item
  Bob can get only one of $m_x$ or $m_z$ because only one bit is
  sent. 
\item
  Due to the algorithm design, Bob is able to figure out the info
  that he wants even if the bit that Alice sends is not the one that
  he wanted.
\item State for \ket{\psi} is carefully chosen for this.
\end{itemize}

Conclusion:
\begin{itemize}
\item This is a crude protocol because Bob cannot know with
  probability 1 whether he got the right bit or not (due to the nature
  of quantum algorithms)
\item Usual way to get around this is to repeat the procedure multiple
  times, but we can't do that here.
\item DBT: Why can't we repeat?? Because measure $\Rightarrow$ state collapsed??
\end{itemize}

\section{BB84 QKD}
\begin{itemize}
\item The first QKD (Quantum Key Distribution) protocol
\item For secret key transmission for use in symmetric key cryptography
\item
  Public key cryptography like RSA = computationally expensive when
  compared to symmetric
  \begin{itemize}
  \item DBT: Even https uses public key cryptography to establish a secret
    before shifting to symmetric key cryptography??
  \end{itemize}
\end{itemize}

Aim:
\begin{itemize}
\item Send secret without an eavesdropper being able to intercept it successfully.
\item
  DBT: Maybe the secret needs to be transmitted over an insecure
  channel ??
\end{itemize}

\subsection{Algorithm}
Alice: 
\begin{itemize}
\item $i \in \{1 \cdots n\}$
\item Pick $b_i \in_{rand} \{X, Z\}$ (ie, choose a basis)
\item Pick $x_i \in_{rand} \{0, 1\}$ (message bit ??)
\item Pick initial state $\ket{\psi}$ as earlier
  \begin{mathpar}
    \ket{\psi} = 
    \begin{cases}
      \ket{+} & \text{if $a=x$, $m_x=0$} \\ 
      \ket{-} & \text{if $a=x$, $m_x=1$} \\ 
      \ket{0} & \text{if $a=z$, $m_x=0$} \\ 
      \ket{1} & \text{if $a=z$, $m_x=1$} \\ 
    \end{cases} \\
\ket{\psi} = \bigotimes\limits_{i=1}^{n} \ket{\psi_i}
  \end{mathpar}
\item This $\ket{\psi}$ is sent to Bob
\end{itemize}

Bob: 
\begin{itemize}
\item $i \in \{1 \cdots n\}$
\item Pick $\tilde{b_i} \in_{rand} \{X, Z\}$ (ie, choose a basis)
\item Measure $\ket{\psi_i}$ collapses to $\tilde{x_i}$
\item $\tilde{x_i} = x_i$ with probability $\frac{1}{2}$
\end{itemize}

Complication-1:
\begin{itemize}
\item Entire $\tilde{x_i}$ cannot be the secret being sent
\item Because it will reach Bob only with probability $\frac{1}{2}$
\item Only half will arrive at Bob on an average
\end{itemize}

Solution:
\begin{itemize}
\item Alice and Bob publicly announce their chosen bases:
  $(b_1, b_2, \cdots)$ and $(\tilde{b_1}, \tilde{b_2}, \cdots)$
\item Drop $x_i$ where $b_i \neq \tilde{b_i}$
\item DBT: How does this make sense ??
\item Because if $b_i = \tilde{b_i}$ Bob will get the message anyway
\item Roughly $\frac{n}{2}$ bits left intact
\item This can become the secret being sent
\item Secret: $x=\tilde{x}$
  \begin{itemize}
  \item Alice: $x$
  \item Bob: $\tilde{x}$
  \end{itemize}
\end{itemize}

Preventing eavesdropping (by Eve):
\begin{itemize}
\item DBT: If Eve measures the info being sent by Alice before Bob does,
  the state would get disturbed and Bob can detect it ??
\item ie, eavesdropping by an adversary causes errors in the leftover
  $x = \tilde{x}$
\end{itemize}

Solution:
\begin{itemize}
\item QBER is done publicly
\item BER: Bit Error Rate
\item Quantum version is QBER: Quantum BER
\item Alice publicly announce a subset of the leftover $x$
  \begin{itemize}
  \item DBT: Only Alice announces?
  \item DBT: And is it just bit position or bit value as well?
  \end{itemize}
\item Both Alice and Bob publicly calculate QBER on these bits
\item Error free $\Rightarrow$ QBER = 0 
  \begin{itemize}
  \item QBER = 0 is the ideal situation
  \item Real life $\Rightarrow$ there always will be error
  \item Efforts are being made to have good bit-rate of data
    transmission while keeping error rate low enough
  \item Eg: IITD-DRDO announced being able to do QKD
    \footnote{\url{https://www.thehindu.com/sci-tech/science/drdo-and-iit-delhi-demonstrate-free-space-quantum-secure-communication-over-1-km-distance/article69701174.ece}}
    \begin{itemize}
    \item 1 km secure communication
    \item Free space (has some advantage over other techniques)
    \item QBER < 7\%
    \item Bitrate: 240 bps 
    \end{itemize}
  \end{itemize}
\end{itemize}

There is more:
\begin{itemize}
\item Error correction
\item DBT: Privacy amplification
\item DBT: Can't Eve impersonate Bob to Alice? At the very beginning? 
\end{itemize}

About RSA:
\begin{itemize}
\item RSA $\Rightarrow$ assumption is that factoring is a hard problem
\item But this is not yet proven
\item Anyway, quantum computer capable of Shor's algorithm renders RSA moot
\item So we need algorithms like this
\end{itemize}

\section{Trivia}
\begin{itemize}
\item
  DBT:
  Research on quantum secure communication predates major interest in
  quantum computing algorithms ??
\item Quantum money
\item One time pad:
  \begin{itemize}
  \item Random number chosen shared between both parties = secret
  \item ciphertext = message xor secret
  \item message = ciphertext xor secret
  \end{itemize}
\end{itemize}

\end{document}
