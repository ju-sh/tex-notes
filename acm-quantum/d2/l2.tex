\documentclass[12pt]{article}

\usepackage{amsmath}
\usepackage{amssymb}
\usepackage{stmaryrd}
\usepackage{booktabs}
\usepackage{mathpartir}
\usepackage[backend=bibtex, style=numeric]{biblatex}
\usepackage{hyperref}
\usepackage{soul}

% https://tex.stackexchange.com/questions/611251/in-circuitikz-multiwire-how-to-so-specify-label-placement
\usepackage[siunitx, RPvoltages, fulldiode, american voltages]{circuitikz}
\newcommand{\mwire}[3]{% args: position (0.1-0.9), label, final coordinate
    coordinate(tmp1) #3 coordinate(tmp2)
    (tmp1) -- ($(tmp1)!{#1-0.1}!(tmp2)$) to[multiwire=#2] ($(tmp1)!{#1+0.1}!(tmp2)$) -- (tmp2)
}

\usepackage{tikz}
\usetikzlibrary{positioning}
\usetikzlibrary{calc,decorations.markings}
\usetikzlibrary{quantikz2}
\usetikzlibrary{external}
\tikzexternalize

\usetikzlibrary{positioning}
\ctikzset{logic ports=ieee}

% \newcommand\ket[1]{
%   | #1 \rangle
% }

%   % \left\langle #1 \middle| #2 \right\rangle
% \newcommand\bra[1]{
%   \langle #1 |
% }

\newcommand\kket[1]{
  | #1 \rangle \rangle
}
% | #1 \rangle \kern-1.35pt \rangle


\begin{document}

\title{Lecture 2: Qubits, quantum states, and quantum gates}
\author{Lecturer: Dinesh Krishnamoorthy\\ Scribe: Julin Shaji}
\date{June 10, 2025}
\maketitle

\begin{itemize}
\item Quantum registers
  \begin{itemize}
  \item Doesn't hold just 0 or 1
  \item Stores \emph{qubits}
  \end{itemize}
\item
  A qubit value is like $\ket{\phi} = \alpha \ket{0} + \beta \ket{1}$
  \begin{itemize}
  \item State can be $0$ with probability $|\alpha|^2$
  \item State can be $1$ with probability $|\beta|^2$
  \item $\alpha, \beta \in \mathbb{C}$ (complex numbers)
  \item For simplicity, we treat $\alpha$ and $\beta$ as $\mathbb{R}$ values for now.
  \item Note: $\alpha$ and $\beta$ can be negative.
  \item Property: $|\alpha|^2 + |\beta|^2 = 1$
  \item DBT: What does $\alpha$ or $\beta$ being negative signify/mean?
  \item $\alpha$ and $\beta$ are \emph{amplitudes}
  \item Analogous values are probabilities in the case of deterministic randomized context.
  \end{itemize}
\item NQ: Why use squares of amplitudes instead of using amplitudes themselves?
  \begin{itemize}
    \item ANS: It corresponds to Euclidean length of vectors
    \item Probably other nice-to-have properties comes for free along
      with this.
  \end{itemize}
\item 
  $\alpha \ket{0} + \beta \ket{1}$ can be represented as the vector%
  \begin{mathpar}%
    \begin{bmatrix}
      \alpha \\
      \beta  \\
    \end{bmatrix}
  \end{mathpar}
\item Current state is a \emph{superposition} of states $\ket{0}$ and $\ket{1}$.
\end{itemize}

Ket notation
\begin{itemize}
\item Popularized by Paul Dirac \cite{dirac1939new}
\item aka Dirac notation
\item 
  $\alpha \ket{0} + \beta \ket{1}$ is pronounced
  'alpha ket zero plus beta ket one'
\item Example of a 2-qubit quantum state:
  \begin{mathpar}
  \frac{1}{2}\ket{00} +
  \frac{1}{2}\ket{01} +
  \frac{1}{2}\ket{10} -
  \frac{1}{2}\ket{11}
   = 
  \begin{bmatrix}
    1/2 \\
    1/2 \\
    1/2 \\
    -1/2 \\
  \end{bmatrix}
  \end{mathpar}
\item Sum of squares of amplitudes = 1 for a valid quantum state
\item The negative sign might as well have been positive in the above example.
\item States in a 2-qubit system:
  \begin{mathpar}
   \ket{00} =%
\begin{bmatrix}
  1 \\
  0 \\
  0 \\
  0 \\
\end{bmatrix}

   \ket{01} =%
\begin{bmatrix}
  0 \\
  1 \\
  0 \\
  0 \\
\end{bmatrix}

   \ket{10} =%
\begin{bmatrix}
  0 \\
  0 \\
  1 \\
  0 \\
\end{bmatrix}

   \ket{11} =%
\begin{bmatrix}
  0 \\
  0 \\
  0 \\
  1 \\
\end{bmatrix}
  \end{mathpar}
\item States in an n-qubit system $\in {01}^n$
    % \sum\limits‎_‎{x \in {\{01\}}^n}^{‎}
  \begin{mathpar}
    \ket{\phi} = \sum_‎{x \in {\{01\}}^n}^{} \alpha_x \ket{x} \\
    \left( \sum_‎{x \in {\{01\}}^n}^{} |\alpha_x|^2 \right) = 1 \\
  \end{mathpar}
\item
  DBT: For a 2-qubit system, the set $\{\ket{00}, \ket{01}, \ket{10},
  \ket{11}\}$ may be thought of as the basis states or the generating
  set ??
\end{itemize}

Double slit experiment (Physics):
\begin{itemize}
\item Light particles exhibiting behaviour of waves
\item Discoveries regarding dual nature of light:
  \begin{itemize}
  \item J. J. Thompson: $e^-$ is particle
    \begin{itemize}
    \item Physics Nobel 1906: For conduction of electricity in gases
    \end{itemize}
  \item de Broglie: $e^-$ is wave
    \begin{itemize}
    \item Confirmed by Davisson-Germer experiment
    \item Physics Nobel 1929
    \end{itemize}
  \end{itemize}
\item
  Electrons could be thought as being a in a superposition of wave
  and particle states.
\end{itemize}

\$-gate:
\begin{itemize}
\item 
  Dirac notation:
  $\frac{1}{\sqrt{2}}\ket{0} +
   \frac{1}{\sqrt{2}}\ket{1}$
\item 
\end{itemize}

\section{'\$-gate'}
  \begin{mathpar}
    \begin{bmatrix}
      1/2 & 1/2 \\
      1/2 & 1/2 \\
    \end{bmatrix}
  \end{mathpar}
\begin{itemize}
\item
  A name we just made up to call the non-quantum but randomized version of
  Hadamard gate
\item Can be thought of as a gate that gives 0 or 1 with equal probability
\item Regardless of the input
\end{itemize}

\$-gate is not a quantum gate. Let's consider an example.

\begin{mathpar}
\begin{bmatrix}
  1 \\  
  0 \\  
\end{bmatrix}
\end{mathpar}
is a valid quantum state because $|1|^2 + |0|^2 = 1$.

Yet applying \$-gate to this state gives

\begin{mathpar}
  \begin{bmatrix}
    1/2 & 1/2 \\
    1/2 & 1/2 \\
  \end{bmatrix}
  \begin{bmatrix}
    1 \\  
    0 \\  
  \end{bmatrix} =
  \begin{bmatrix}
    1/2 \\  
    1/2 \\  
  \end{bmatrix}
\end{mathpar}

which is not a quantum state since 
$|1/2|^2 + |1/2|^2 = 1/2 \neq 1$.


\section{Reversible computations}
\begin{itemize}
\item Any reversible computation is a \emph{permutation matrix} 
  \begin{itemize}
  \item Every row has exactly one 1
  \item Every column has exactly one 1
  \end{itemize}
\item Reversible $\Rightarrow$ it's just a rearrangement
\item Classical reversible operations are also quantum operations ??
\item
  In order to be reversible, we need to be able to 'unflip the
  coin' in the case of a coin toss.
\end{itemize}

\section{Measurement}
\begin{itemize}
\item A quantum system is in a superposition of states
\item Measure it $\Rightarrow$ it collapses to one of the superpositioned states
\end{itemize}

\section{Building quantum chips}
\begin{itemize}
\item DBT: Needs Hadamard gates ?? As a building block ??
\item
  Handling of $\mathbb{R}$ values needs arbitrary precision, which
  is not possible. So we can deal with an accuracy $\epsilon$ instead.
\item Solovay-Kitaev theorem. Mentions number of quantum gates needed ??
  \begin{mathpar}
    1000t\left(log \frac{t}{\epsilon} \right)
  \end{mathpar}
  where $t$ is the number of quantum gates ??
\end{itemize}


\section{Quantum gates}
\subsection{Properties of quantum gates}
\begin{itemize}
\item Transforms a qubit
  \begin{itemize}
  \item Input = a qubit
  \item Output = a qubit
  \end{itemize}
\item Must be a linear transformation
  \begin{itemize}
  \item Scaling the whole thing must be equal to scaling the
    individual components and then adding them up
  \item
    $T(\alpha \ket{u} + \beta \ket{v}) =
    \alpha T(\ket{u}) + \beta T(\ket{v})$
  \end{itemize}
  \begin{enumerate}
  \item $T(\ket{u} + \ket{v}) = T(\ket{u}) + T(\ket{v})$
  \item $T(\alpha \ket{u}) = \alpha T(\ket{u})$
  \end{enumerate}
\end{itemize}

\subsection{Hadamard gate}

\begin{center}
  \begin{quantikz}
   \lstick{x} & \gate{H} & \rstick{y} \\
  \end{quantikz}

  \begin{mathpar}
    \begin{bmatrix}
      1/\sqrt{2} & 1/\sqrt{2} \\
      1/\sqrt{2} & -1/\sqrt{2} \\
    \end{bmatrix}
  \end{mathpar}
\end{center}

\begin{itemize}
\item Acts on a single qubit
\item DBT: Creates superposition of input basis states ?? $^w$ \cite{hadamardgate}
  \begin{mathpar}
  \ket{0} \mapsto \frac{\ket{0}+\ket{1}}{\sqrt{2}}

  \ket{1} \mapsto \frac{\ket{0}-\ket{1}}{\sqrt{2}}
  \end{mathpar}
\end{itemize}

The operation can be represented as a matrix $\Rightarrow$ it's linear.

But does it make valid qubits? Let's try an example with a vector.

\begin{mathpar}
  \begin{bmatrix}
    1/\sqrt{2} & 1/\sqrt{2} \\
    1/\sqrt{2} & -1/\sqrt{2} \\
  \end{bmatrix}
  \begin{bmatrix}
    \alpha \\      
    \beta  \\      
  \end{bmatrix} =
  \begin{bmatrix}
    \alpha/\sqrt{2} + \beta/\sqrt{2} \\
    \alpha/\sqrt{2} - \beta/\sqrt{2} \\
  \end{bmatrix}
  
 \left(\frac{\alpha + \beta}{\sqrt{2}}\right)^2 +
 \left(\frac{\alpha - \beta}{\sqrt{2}}\right)^2 =
 |\alpha|^2 + |\beta|^2 = 1
\end{mathpar}

So valid.

Hadamard gate is reversible. What happens when we apply Hadamard gate
twice to an input?

% \begin{quantikz}[slice all]
%  \lstick{$\ket{0}$} & \gate{H} & \gate{H} & \meter{}
% \end{quantikz}
\begin{quantikz}
   \lstick{$\ket{0}$}
 \slice[label style={pos=1, anchor=north}]{$\ket{0}$}
& 
 \gate{H}
 \slice{\tiny $\frac{1}{\sqrt{2}}\ket{0} + \frac{1}{\sqrt{2}}\ket{1}$} 
 &
 \gate{H}
 \slice[label style={pos=1, anchor=north}]{\tiny $H\left(
     \frac{1}{\sqrt{2}}\ket{0} + \frac{1}{\sqrt{2}}\ket{1} \right)$} 
 & \meter{}
\end{quantikz}
 %\lstick{$\ket{0}$}\slice{step} & \gate{H}\slice{step} & \gate{H}\slice{step} & \meter{} \\

Let's use linearity instead of explicitly expanding $H$.

\begin{mathpar}
  H\left(\frac{1}{\sqrt{2}}\ket{0} + \frac{1}{\sqrt{2}}\ket{1} \right) \\
= \frac{1}{\sqrt{2}}H\ket{0} + \frac{1}{\sqrt{2}}H\ket{1} \\
= \frac{1}{\sqrt{2}}\left(\frac{1}{\sqrt{2}}\ket{0} + \frac{1}{\sqrt{2}}\ket{1} \right)
  +
  \frac{1}{\sqrt{2}}\left(\frac{1}{\sqrt{2}}\ket{0} -
    \frac{1}{\sqrt{2}}\ket{1} \right) \\
= \frac{1}{2}\ket{0} + 
  \frac{1}{2}\ket{1} + 
  \frac{1}{2}\ket{0} - 
  \frac{1}{2}\ket{1} \\
= \ket{0}   
\end{mathpar}

ie, starting with $\ket{0}$, we \emph{always} get $\ket{0}$ itself
upon applying a Hadamard gate.

There is a destructive interference happening. Which is why two terms
got cancelled out in the above derivation.

\subsection{Swap gate}
\begin{itemize}
\item $(x,y) \mapsto (y,x)$
\item Classical version:
  \begin{mathpar}
    \begin{bmatrix}
      1 & 0 & 0 & 0 \\     
      0 & 0 & 1 & 0 \\     
      0 & 1 & 0 & 0 \\     
      0 & 0 & 0 & 1 \\     
    \end{bmatrix}
  \end{mathpar}
\end{itemize}

\subsection{Bra-ket notation}
\begin{itemize}
\item Column vectors are represented with ket.
  \begin{mathpar}
    \ket{00} =
    \begin{bmatrix}
       1 \\ 
       0 \\ 
       0 \\ 
       0 \\ 
    \end{bmatrix}
  \end{mathpar}
\item Row vectors are represented with bra.
  \begin{mathpar}
    \bra{00} =
    \begin{bmatrix}
       1 & 0 & 0 & 0 
    \end{bmatrix}
  \end{mathpar}
\end{itemize}

Let's us try to use the bra-ket notation to write the swap gate.

\begin{mathpar}
  \begin{bmatrix}
    1 & 0 & 0 & 0 \\     
    0 & 0 & 1 & 0 \\     
    0 & 1 & 0 & 0 \\     
    0 & 0 & 0 & 1 \\     
  \end{bmatrix} =
  \begin{bmatrix}
    1 & 0 & 0 & 0 \\     
    0 & 0 & 0 & 0 \\     
    0 & 0 & 0 & 0 \\     
    0 & 0 & 0 & 0 \\     
  \end{bmatrix} + 
  \begin{bmatrix}
    0 & 0 & 0 & 0 \\     
    0 & 0 & 1 & 0 \\     
    0 & 0 & 0 & 0 \\     
    0 & 0 & 0 & 0 \\     
  \end{bmatrix} +
  \begin{bmatrix}
    0 & 0 & 0 & 0 \\     
    0 & 0 & 0 & 0 \\     
    0 & 1 & 0 & 0 \\     
    0 & 0 & 0 & 0 \\     
  \end{bmatrix} +
  \begin{bmatrix}
    0 & 0 & 0 & 0 \\     
    0 & 0 & 0 & 0 \\     
    0 & 0 & 0 & 0 \\     
    0 & 0 & 0 & 1 \\     
  \end{bmatrix}
  
= 
  \begin{bmatrix}
     1 \\ 
     0 \\ 
     0 \\ 
     0 \\ 
  \end{bmatrix}
  \begin{bmatrix}
     1 & 0 & 0 & 0 
  \end{bmatrix} + 
  \begin{bmatrix}
     0 \\ 
     1 \\ 
     0 \\ 
     0 \\ 
  \end{bmatrix}
  \begin{bmatrix}
     0 & 0 & 1 & 0 
  \end{bmatrix} + 
  \begin{bmatrix}
     0 \\ 
     0 \\ 
     1 \\ 
     0 \\ 
  \end{bmatrix}
  \begin{bmatrix}
     0 & 1 & 0 & 0 
  \end{bmatrix} + 
  \begin{bmatrix}
     0 \\ 
     0 \\ 
     0 \\ 
     1 \\ 
  \end{bmatrix}
  \begin{bmatrix}
     0 & 0 & 0 & 1 
  \end{bmatrix}
  
= \ket{00}\bra{00} +
  \ket{01}\bra{10} +
  \ket{10}\bra{01} +
  \ket{11}\bra{11}
\end{mathpar}


Now, let us try the Hadamard gate with bra-ket notation.

\begin{mathpar}
H =
\ket{0}\left(\frac{1}{\sqrt{2}}\bra{0} + \frac{1}{\sqrt{2}}\bra{1}\right) +
\ket{1}\left(\frac{1}{\sqrt{2}}\bra{0} - \frac{1}{\sqrt{2}}\bra{1}\right)
\end{mathpar}

