\documentclass[12pt]{article}

\usepackage{amsmath}
\usepackage{amssymb}
\usepackage{stmaryrd}
\usepackage{booktabs}
\usepackage{mathpartir}
\usepackage[backend=bibtex, style=numeric]{biblatex}
\usepackage{hyperref}
\usepackage{soul}

% https://tex.stackexchange.com/questions/611251/in-circuitikz-multiwire-how-to-so-specify-label-placement
\usepackage[siunitx, RPvoltages, fulldiode, american voltages]{circuitikz}
\newcommand{\mwire}[3]{% args: position (0.1-0.9), label, final coordinate
    coordinate(tmp1) #3 coordinate(tmp2)
    (tmp1) -- ($(tmp1)!{#1-0.1}!(tmp2)$) to[multiwire=#2] ($(tmp1)!{#1+0.1}!(tmp2)$) -- (tmp2)
}

\usepackage{tikz}
\usetikzlibrary{positioning}
\usetikzlibrary{calc,decorations.markings}
\usetikzlibrary{quantikz2}
\usetikzlibrary{external}
\tikzexternalize

\usetikzlibrary{positioning}
\ctikzset{logic ports=ieee}

% \newcommand\ket[1]{
%   | #1 \rangle
% }

%   % \left\langle #1 \middle| #2 \right\rangle
% \newcommand\bra[1]{
%   \langle #1 |
% }

\newcommand\kket[1]{
  | #1 \rangle \rangle
}
% | #1 \rangle \kern-1.35pt \rangle


\begin{document}

\title{Lecture 1: Classical computation through circuits}
\author{Lecturer: Jaikumar Radhakrishnan\\ Scribe: Julin Shaji}
\date{June 09, 2025}
\maketitle

\section{Register states}
\begin{itemize}
\item
  One register can hold one bit
\item
  State of a 1-bit register $\in \{0,1\}$
\item
  State of a 2-bit register $\in \{0,1\}^2 = \{00,01,10,11\}$
\item
  State of an n-bit register $\in \{0,1\}^n$. ie, set of all n-length
  binary strings
\item 
  One hot encoding $\Rightarrow$ n-bit state can represented by a
  vector of length $2^n$
  
  \begin{mathpar}
   00 \Rightarrow%
\begin{bmatrix}
  1 \\
  0 \\
  0 \\
  0 \\
\end{bmatrix}

   01 \Rightarrow%
\begin{bmatrix}
  0 \\
  1 \\
  0 \\
  0 \\
\end{bmatrix}

   10 \Rightarrow%
\begin{bmatrix}
  0 \\
  0 \\
  1 \\
  0 \\
\end{bmatrix}

   11 \Rightarrow%
\begin{bmatrix}
  0 \\
  0 \\
  0 \\
  1 \\
\end{bmatrix}
  \end{mathpar}
\end{itemize}

\section{Representing functions}
\subsection{Truth table}
Any deterministic Boolean function can be represented by means of its
truth table.

Example: AND operation

% \begin{center}
\begin{tabular}{cc|c}
  \toprule
  A & B & C \\
  \midrule
  0 & 0 & 0 \\
  0 & 1 & 0 \\
  1 & 0 & 0 \\
  1 & 1 & 1 \\
  \bottomrule
\end{tabular}
% \end{center}
%
\begin{mathpar}
  \begin{bmatrix}
     1 & 1 & 1 & 0 \\ 
     0 & 0 & 0 & 1 \\ 
  \end{bmatrix}
\end{mathpar}

% Self-check-1
Even parity gate (true if parity is even, essentially XNOR gate):

\begin{center}
\begin{tabular}{cc|c}
  \toprule
  A & B & C \\
  \midrule
  0 & 0 & 1 \\
  0 & 1 & 0 \\
  1 & 0 & 0 \\
  1 & 1 & 1 \\
  \bottomrule
\end{tabular}
\end{center}

\begin{mathpar}
M_{T_1} =
  \begin{bmatrix}
     0 & 1 & 1 & 0 \\ 
     1 & 0 & 0 & 1 \\ 
  \end{bmatrix}

M_{T_2} =
  \begin{bmatrix}
     1 & 0 \\ 
     0 & 1 \\ 
  \end{bmatrix}

M_{T_3} =
  \begin{bmatrix}
     0 & 0 & 1 & 0 & 1 & 0 & 0 & 0 \\ 
     0 & 0 & 0 & 1 & 0 & 1 & 0 & 0 \\ 
     1 & 0 & 0 & 0 & 0 & 0 & 1 & 0 \\ 
     0 & 1 & 0 & 0 & 0 & 0 & 0 & 1 \\ 
  \end{bmatrix}
\end{mathpar}

\section{Kronecker product}
\begin{itemize}
\item
  $A \otimes B$
\item
  Named after German mathematician Leopold Kronecker (1823-1891).
\item 
  Take each element of $A$ and multiply it with the entire $B$. The
  resultant matrix will take the position of that element in $A$.
\item
  Oversimplification:
  Think of it as a copy of $B$ being inserted wherever the element
  is non-zero in $A$
\end{itemize}

\section{Sequential composition}
\begin{itemize}
\item Operation 1: $M_1$
\item Operation 2: $M_2$
\item Operation 1, then operation 2: $M_2 \cdot M_1$
\end{itemize}

\section{Parallel operations}
\begin{itemize}
\item Operation 1: $M_1$
\item Operation 2: $M_2$
\item Operation 1 and operation 2: $M_1 \otimes M_2$
\end{itemize}

\section{Misc}
\subsection{Hadamard product}
\begin{itemize}
\item Element-wise matrix multiplication
\item Both operand matrices must be of the same shape
\item Like matrix addition but with multiplication as the operation
\end{itemize} 

\subsection{Fourier transform}
\begin{itemize}
\item Converts a function in time domain to frequency domain 
\end{itemize}

\subsection{Hadamard transform}
\begin{itemize}
\item Has a crucial role in quantum computing
\item A generalized form of Fourier transform
\item By Jacques Hadamard, Hans Rademacher, Joseph Walsh
\item Is a \emph{Hadamard matrix} $H_m$ of shape $2^m \times 2^m$
\item Excluding the normalization factor, $H_m$ consists entirely of $1$ and $-1$
\end{itemize}

Recursive definition:

\begin{mathpar}
 H_0 = 1 
 
H_m = \frac{1}{\sqrt{2}}%
\begin{pmatrix}
  H_{m-1} & H_{m-1} \\
  H_{m-1} & -H_{m-1} \\
\end{pmatrix}
\end{mathpar}

\subsection{Toffoli gate}
\begin{mathpar}
Toff(x, y, z) = (x, y, z \oplus xy)
\end{mathpar}

\begin{itemize}
\item AND(x,y) = Toff(x, y, x) = Toff(x, y, y)
\item XOR(x,y) = Toff(1, y, x)
\end{itemize}

\subsection{More}
\begin{itemize}
\item Toffoli gate and Fredkin gate: reversible universal gates
\item
  DBT: How to use parallelize OR-AND?\\
  The matrix would be $4 \times 16$.\\
  How would input vector look like?\\
  ANS: 4 inputs $\Rightarrow 2^4 \Rightarrow$ state space can be
  represented via one-hot-encoding as a vector of length 16. 
  
  DBT: So a translation is needed? From two 4-size vectors to a
  single 16-size vector?
\end{itemize}
