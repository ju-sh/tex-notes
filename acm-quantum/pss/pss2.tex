\documentclass[12pt]{exam}

\usepackage{amsmath}
\usepackage{amssymb}
\usepackage{stmaryrd}
\usepackage{booktabs}
\usepackage{mathpartir}
\usepackage[backend=bibtex, style=numeric]{biblatex}
\usepackage{hyperref}
\usepackage{soul}

% https://tex.stackexchange.com/questions/611251/in-circuitikz-multiwire-how-to-so-specify-label-placement
\usepackage[siunitx, RPvoltages, fulldiode, american voltages]{circuitikz}
\newcommand{\mwire}[3]{% args: position (0.1-0.9), label, final coordinate
    coordinate(tmp1) #3 coordinate(tmp2)
    (tmp1) -- ($(tmp1)!{#1-0.1}!(tmp2)$) to[multiwire=#2] ($(tmp1)!{#1+0.1}!(tmp2)$) -- (tmp2)
}

\usepackage{tikz}
\usetikzlibrary{positioning}
\usetikzlibrary{calc,decorations.markings}
\usetikzlibrary{quantikz2}
\usetikzlibrary{external}
\tikzexternalize

\usetikzlibrary{positioning}
\ctikzset{logic ports=ieee}

% \newcommand\ket[1]{
%   | #1 \rangle
% }

%   % \left\langle #1 \middle| #2 \right\rangle
% \newcommand\bra[1]{
%   \langle #1 |
% }

\newcommand\kket[1]{
  | #1 \rangle \rangle
}
% | #1 \rangle \kern-1.35pt \rangle


\title{Problem solving session 2}
\author{\tiny{JS}}
\date{16 June 2025}

\begin{document}

\maketitle
\printanswers


Given two primes numbers $p$, $q$, let $N = pq$.

\begin{questions}

\question
Express $\phi(N)$ in terms of $p$ and $q$.

\begin{solution}
  \begin{align*}
\phi(N) &= \phi(pq) \\
        &= \phi(p) \cdot \phi(q) \\
        &= (p-1)(q-1) \\
  \end{align*}
  
Because:
\begin{itemize}
\item $\phi(n) = n-1$ when $x$ is prime.
\item Chinese remainder theorem
\end{itemize}

\end{solution}

\question
Suppose $N$ as described above and an $e < \phi(N)$ is given with
$gcd(e, \phi(N)) = 1$.
Let $d$ be an integer such that

\begin{mathpar}
ed \equiv 1 \pmod{\phi(N)}
\end{mathpar}

Argue that such a $1 \le d < \phi(N)$ must always exist.
Also, give an algorithm that given $ϕ(N)$ and $e$ can compute such a $d$.

\begin{solution}
  
\end{solution}

\question
The factoring problem is as follows: given a composite integer $N$, obtain two
integers $a$, $b$ such that $N = ab$ with both $a$ and $b$ more than
$1$.
Suppose that we have a (black box) access to an algorithm for
factoring.

Let $m \in \mathbb{Z}_N^∗$
Suppose that you are given $(N, e)$ and a $c$ which is generated by computing
$m^e \pmod{N}$.

Argue that using a black box algorithm for factoring, from $(N, e)$
and $c$ it is possible to recover $m$.

\begin{solution}
  
\end{solution}

\end{questions}
\end{document}
