\documentclass[12pt]{article}

\usepackage{amsmath}
\usepackage{amssymb}
\usepackage{stmaryrd}
\usepackage{booktabs}
\usepackage{mathpartir}
\usepackage[backend=bibtex, style=numeric]{biblatex}
\usepackage{hyperref}
\usepackage{soul}

% https://tex.stackexchange.com/questions/611251/in-circuitikz-multiwire-how-to-so-specify-label-placement
\usepackage[siunitx, RPvoltages, fulldiode, american voltages]{circuitikz}
\newcommand{\mwire}[3]{% args: position (0.1-0.9), label, final coordinate
    coordinate(tmp1) #3 coordinate(tmp2)
    (tmp1) -- ($(tmp1)!{#1-0.1}!(tmp2)$) to[multiwire=#2] ($(tmp1)!{#1+0.1}!(tmp2)$) -- (tmp2)
}

\usepackage{tikz}
\usetikzlibrary{positioning}
\usetikzlibrary{calc,decorations.markings}
\usetikzlibrary{quantikz2}
\usetikzlibrary{external}
\tikzexternalize

\usetikzlibrary{positioning}
\ctikzset{logic ports=ieee}

% \newcommand\ket[1]{
%   | #1 \rangle
% }

%   % \left\langle #1 \middle| #2 \right\rangle
% \newcommand\bra[1]{
%   \langle #1 |
% }

\newcommand\kket[1]{
  | #1 \rangle \rangle
}
% | #1 \rangle \kern-1.35pt \rangle


\begin{document}

\title{Lecture 2: Quantum Search}
\author{Lecturer: Rajat Mittal\\ Scribe: Julin Shaji}
\date{June 12, 2025}
\maketitle

Notes: \url{https://acm-qc-summer-school.gitlab.io/assets/pdfs/grover.pdf}

Also see: \url{https://youtu.be/hnpjC8WQVrQ}

\section{Grover's algorithm}
\begin{itemize}
\item A quantum algorithm for searching.
\item Grover database search algorithm~\cite{grover1996fast} (1996)
\item
  Finds with high probability input for a particular output wrt to
  a function $f$
\item
  Needs only O($\sqrt(n)$) evaluations of $f$, where $n$ is size
  of $Domain(f)$
\item Best classical algorithm => $\Theta(n)$ time
\item Grover's algorithm => O($\sqrt(n)$)
\item A quadratic speedup.
  \begin{itemize}
  \item
    Not an exponential speedup, but still quite significant for
    large $n$
  \item
    Eg: A 128-bit key used in symmetric cryptography could be
    brute forced with $2^{64}$ iterations 
  \end{itemize}
\item Algorithm created by Lov Grover
  \begin{itemize}
  \item He has Indian roots
  \item Born in Meerut, UP
  \item Bachelor's from IIT Delhi
  \end{itemize}
\end{itemize}


\begin{itemize}
\item Input:
  \begin{itemize}
  \item An unstructured database. Think of it as a list $L$ of size $n$.
  \item $L \in \{01\}^n$
  \item Some of the elements of $L$ are 'marked'
  \item $L_i = 1$ if $L_i$ is marked
  \item Randomly distributed values
  \item
    So the best we can do in classical setting is to examine all one
    by one.
  \item There is an oracle $O$ that tells whether a particular element of
    $L$ is marked 
  \item $O(L_i) = 1$ if $L_i$ is marked
  \end{itemize}
\item Output:
  \begin{itemize}
  \item Find the position of a marked element in $L$
  \item With minimal number of queries to the oracle
  \item
    Could be thought of as finding the input for a desired value of
    the function
  \item
    Could be thought of as a decision problem if what we are
    asking is the existence of a marked element.
  \end{itemize}
\end{itemize}

There are many variants of Grover's search algorithm.

About the oracle:

\begin{itemize}
\item Oracle is not the entire function
\item It is just a verifier
\item Eg: In a SAT problem for a CNF formula, oracle can check if a
  given set of values evaluate to true. 
\item 
  Quantum algorithms are helpful to solve NP-Complete problems because
  their solutions are easy to verify $\Rightarrow$ oracle.
\end{itemize}

\subsection{Grover with exactly one marked element}
Assumptions:
\begin{itemize}
\item Number of marked elements = 1
\item $n = 2^k$ which makes the index $i$ a k-bit string
\end{itemize}

$L$ can be thought of as the truth table for a function
$f: \{01\}^k \mapsto {01}$ like this:

\begin{mathpar}
  f(i) =
  \begin{cases}
      1 & L_i \text{ is marked} \\
      0 & L_i \text{ is unmarked} \\
  \end{cases}
\end{mathpar}

What the oracles tells us is:
\begin{itemize}
\item $\ket{i, b} \mapsto \ket{i, b \oplus f(i)}$
\item $i$ is a k-bit string
\item $b$ is a 1-bit string. Let its initial value be $0$
\item This can be made as:
  \begin{mathpar}
O_f(\ket{i, b}) = (-1)^{f(i).b}\ket{i, b}
  \end{mathpar}
   % DBT: -1^(f(i) why mult by b ?
\end{itemize}

\begin{itemize}
\item 
  Given the oracle $O_f$, we need to find an $i$ such that $f(i)$
  would give $1$ with minimum number of queries to oracle.
\item 
  Randomized setting:
  Only option is to keep randomly selecting $L_i$ values till oracle
  says all right.\\
  Needs $\Omega(n)$ queries to oracle.
\item 
  We need to get probability close to constant.
\end{itemize}


\begin{itemize}
\item There is only one marked element, say $x$
\item State of marked elements: $\ket{M} = \ket{x}$
  \begin{itemize}
  \item Because there is only one marked element
  \item If we were to choose an element from the set of marked
    elements, we would choose it with probability 1
  \end{itemize}
\item State of unmarked elements:
  \begin{mathpar}
  \ket{U} = \frac{1}{\sqrt{n}}\sum_{L_i \neq x}\ket{L_i}
  \end{mathpar}
  \begin{itemize}
  \item Because there are $n-1$ unmarked elements
  \item We pick them with probability $\frac{1}{n}$
  \item Therefore amplitude of each component would be $\frac{1}{\sqrt{n}}$
  \end{itemize}
\end{itemize}


\begin{itemize}
\item For a quantum algorithm, we need an initial state.
\item
  Let's make an equal superposition with Walsh-Hadamard transform
  ($H^{\otimes n}$)
\item 
  \begin{mathpar}
    \ket{\psi} = \frac{1}{\sqrt{n}} \sum_{i} \ket{L_i}
  \end{mathpar}
\item At this point, probability of finding $x$ is same as finding any
  other index = $\frac{1}{\sqrt{n}}^2 = \frac{1}{n}$, which is quite small
\item We need to move this probability closer to the state of
  finding a marked element: $\ket{M}$
\item $\ket{U}$ and $\ket{M}$ are orthogonal to each other.
\item Initially, $\ket{\psi}$ is very close to $\ket{U}$ and far from $\ket{M}$
  \begin{mathpar}
    \begin{array}{rcl}
      % \st{%
\ket{\psi} & = & \alpha\ket{M} + \beta\ket{U} \\
           & = & \frac{1}{\sqrt{n}}\ket{M} +
                 \frac{n-1}{\sqrt{n}}\ket{M} \\
      % }
\\
\ket{\psi} & = & cos\theta\ket{M} + sin\theta\ket{U} \\
    \end{array}
  \end{mathpar}
  DBT: But how would we know $\theta$
\end{itemize}

\begin{itemize}
\item
  The oracle returns $-\ket{\psi}$ if $i$ is marked, which we use
  to rotate via reflections to move $\ket{\psi}$ closer to $\ket{M}$.
\item
  If $i$ is for an unmarked element, oracle returns $\ket{\psi}$
  itself, leaving the state unchanged
\item
  The state $\ket{\psi}$ must remain on the unit circle in the
  plane formed by $\ket{M}$ and $\ket{U}$.
\end{itemize}


\end{document}
