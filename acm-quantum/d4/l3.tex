\documentclass[12pt]{article}

\usepackage{amsmath}
\usepackage{amssymb}
\usepackage{stmaryrd}
\usepackage{booktabs}
\usepackage{mathpartir}
\usepackage[backend=bibtex, style=numeric]{biblatex}
\usepackage{hyperref}
\usepackage{soul}

% https://tex.stackexchange.com/questions/611251/in-circuitikz-multiwire-how-to-so-specify-label-placement
\usepackage[siunitx, RPvoltages, fulldiode, american voltages]{circuitikz}
\newcommand{\mwire}[3]{% args: position (0.1-0.9), label, final coordinate
    coordinate(tmp1) #3 coordinate(tmp2)
    (tmp1) -- ($(tmp1)!{#1-0.1}!(tmp2)$) to[multiwire=#2] ($(tmp1)!{#1+0.1}!(tmp2)$) -- (tmp2)
}

\usepackage{tikz}
\usetikzlibrary{positioning}
\usetikzlibrary{calc,decorations.markings}
\usetikzlibrary{quantikz2}
\usetikzlibrary{external}
\tikzexternalize

\usetikzlibrary{positioning}
\ctikzset{logic ports=ieee}

% \newcommand\ket[1]{
%   | #1 \rangle
% }

%   % \left\langle #1 \middle| #2 \right\rangle
% \newcommand\bra[1]{
%   \langle #1 |
% }

\newcommand\kket[1]{
  | #1 \rangle \rangle
}
% | #1 \rangle \kern-1.35pt \rangle


\begin{document}

\title{Lecture 3: Classical probability inequalities}
\author{Lecturer: Jaikumar Radhakrishnan\\ Scribe: Julin Shaji}
\date{June 12, 2025}
\maketitle


\section{Markov's inequality}
\begin{itemize}
\item Given: A non-negative random variable $X$
\item Needed: Probability that $X \ge t$ where $t > 0$
  \begin{itemize}
  \item $P(x \ge t) \le \frac{\mu}{t}$
  \item where $\mu$ is expectation of the random var $X$ (ie, $E[X]$)
  \end{itemize}
\item Example for usefulness: given that average mark of a set of
  students is 5, at most how many student can have 50 marks?
  % not sure about this
\end{itemize}

% Plot graph for step func J and the overlaying linear function

\begin{itemize}
\item P=0 for x<t, otherwise P=1
\item Let this be represented as a function $J$, which is a step function
\item A function $y=\frac{x}{t}$ would completely cover $J$.
\item $E[J] = P(x \ge t)$
\item So,
  \begin{mathpar}
    \begin{array}{rcl}
E[J] & \le & E[\frac{x}{t}]   \\
     & \le & \frac{1}{t} E[x] \\
     & \le & \frac{1}{t} \mu  \\
    \end{array}
  \end{mathpar}
\item Variance = $\sigma^2$ (where $\sigma$ is standard deviation)
  \begin{mathpar}
    \begin{array}{rcl}
Var(x) & = & E[(x-\mu)^2] \\
       & = & \sigma^2     \\
    \end{array}
  \end{mathpar}
\end{itemize}

\section{Chebyshev's inequality}

\begin{itemize}
\item Given: random variable $X$
\item 
\begin{mathpar}
  P(|X - \mu| \ge t) \le \frac{\sigma^2}/{t^2}
\end{mathpar}
\item where $t > 0$
\end{itemize}

% Plot graph here

\begin{itemize}
\item P=0 for $x \in [\mu-t, \mu+t]$ (closed interval)
\item
  Let this be represented as a function $J$, which is like a band-stop
filter function
\item Parabola shape sufficient to completely cover $J$.
  \begin{itemize}
  \item Modulus won't cut it because it would work only for x=1 ??
  \end{itemize}
\item $E[J] = P(|x - \mu| \ge t)$
  \begin{mathpar}
    \begin{array}{crcl}
               &    J & \le & \frac{(x-\mu)^2}{t^2}     \\
  \Rightarrow  & E[J] & \le & E[\frac{(x-\mu)^2}{t^2}]  \\
  \Rightarrow  &      & \le & \frac{1}{t^2}E[(x-\mu)^2] \\
  \Rightarrow  &      & \le & \frac{\sigma^2}{t^2}      \\
    \end{array}
  \end{mathpar}
\end{itemize} 

\section{Sum of n independent 01 variables} 

\begin{itemize}
\item $S = x_1 + x_2 + .... + x_n$
\item The variables are \emph{independent}, so their probabilities are
  independent of each other.
  \begin{mathpar}
    x_i =
    \begin{cases}
      0 & \text{with probability } p \\
      1 & \text{with probability } (1-p)
    \end{cases}
    
E[x_i] = p \\
Var(x_i) = p(1-p) \\     
  \end{mathpar}
\item Since the $x_i$-s are independent (pairwise independence
  would've been sufficient too),
  \begin{mathpar}
    \begin{array}{rcl}
E[S] & = & E[x_1] + E[x_2] + .... + E[x_n] \\
     & = & np \\
    \end{array}
    \begin{array}{rcl}
Var(S) & = & E(x_1) + E(x_2) + .... + E(x_n) \\
       & = & n*p(1-p) \\
    \end{array}
  \end{mathpar}
\end{itemize}

Suppose we have a randomized algorithm $A$.

\begin{itemize}
\item 
\begin{mathpar}
    \begin{cases}
      correct & \text{with probability } \frac{1}{2}+\epsilon \\
      wrong   & \text{with probability } \frac{1}{2}-\epsilon \\
    \end{cases}
\end{mathpar}
\item where $\epsilon$ is a small value
\item For a problem with output $\in \{0, 1\}$
\item
  Getting an algorith which gives 0 or 1 with equal probability
  $\frac{1}{2}$ is easy.
\item
  Because there is no bias. It's just a matter of getting the output
  of a \$-gate. Or flipping a coin.
  (Reminder: We are currently in a randomized setting.)
\item
  Doing better than equal probability, ie having a $\pm \epsilon$ is
  less trivial.
\item But if such a function is available, the small delta will
  accumulate to add up if we repeatedly compute the same thing,
  eventually letting us have more certainity about the output. 
\end{itemize}

% Plot graph for Chebyshev equality


\end{document}
